	
   分散式詢問及監督系統主要被用於分散式資料如檔案或是紀錄的維護上。網路內的使用者可以自系統中查詢所需資料如總和或平均值,若同時將所有原始資料做傳遞及運算,將耗費相當大的網路頻寬及運算資源。於是,內網路聚集技術被提出來降低分散式詢問及監督系統的負擔。然而,這個技術卻容易遭受安全威脅。過去的研究大多假設資料來源為可信任,並針對聚集架構進行安全性的研究。然而,我們認為聚集查詢結果應該在面對惡意攻擊者將錯誤資料置入資料串流進行聚集前,就應該具備強健的容錯能力。傳統上,一個強健的估計值被定義為即使資料來源有誤時,亦能維持一定程度正確性的聚集結果。許多常見的強健估計值是建立在有序統計學上,因此,我們將重心放在內網路計算上之有序統計的可驗證技術。此技術的挑戰為在網路遭受惡意團體介入聚集程序時,仍能確保聚集結果或近似結果的準確性。